\documentclass[11pt,a4paper]{report}

\usepackage[margin=0.6in]{geometry}
\usepackage{titlesec}% http://ctan.org/pkg/titlesec
\usepackage{listings}
\usepackage{color}

\title{Project Euler: Problem Analysis}
\author{Stephen O'Brien}

% Change amount of space after "Chapter" heading (<after-sep> param)
\titleformat{\chapter}{\Huge\bfseries}{\chaptername\ \thechapter}{0pt}{\vskip 20pt\raggedright}%
\titlespacing{\chapter}{0pt}{50pt}{10pt}% \titlespacing{<command>}{<left>}{<before-sep>}{<after-sep>}[<right>]

% Increase table row height globally
\renewcommand{\arraystretch}{2}

\definecolor{gray_ulisses}{gray}{0.55}
\definecolor{castanho_ulisses}{rgb}{0.71,0.33,0.14}
\definecolor{preto_ulisses}{rgb}{0.41,0.20,0.04}
\definecolor{green_ulises}{rgb}{0.2,0.75,0}

% Haskell: for syntax highlighting
\lstdefinelanguage{HaskellUlisses} {
	basicstyle=\ttfamily\scriptsize,
	sensitive=true,
	morecomment=[l][\color{gray_ulisses}\ttfamily\scriptsize]{--},
	morecomment=[s][\color{gray_ulisses}\ttfamily\scriptsize]{\{-}{-\}},
	morestring=[b]",
	stringstyle=\color{red},
	showstringspaces=false,
	numberstyle=\tiny,
	numberblanklines=true,
	showspaces=false,
	breaklines=true,
	showtabs=false,
	emph=
	{[1]
		FilePath,IOError,abs,acos,acosh,all,and,any,appendFile,approxRational,asTypeOf,asin,
		asinh,atan,atan2,atanh,basicIORun,break,catch,ceiling,chr,compare,concat,concatMap,
		const,cos,cosh,curry,cycle,decodeFloat,denominator,digitToInt,div,divMod,drop,
		dropWhile,either,elem,encodeFloat,enumFrom,enumFromThen,enumFromThenTo,enumFromTo,
		error,even,exp,exponent,fail,filter,flip,floatDigits,floatRadix,floatRange,floor,
		fmap,foldl,foldl1,foldr,foldr1,fromDouble,fromEnum,fromInt,fromInteger,fromIntegral,
		fromRational,fst,gcd,getChar,getContents,getLine,head,id,inRange,index,init,intToDigit,
		interact,ioError,isAlpha,isAlphaNum,isAscii,isControl,isDenormalized,isDigit,isHexDigit,
		isIEEE,isInfinite,isLower,isNaN,isNegativeZero,isOctDigit,isPrint,isSpace,isUpper,iterate,
		last,lcm,length,lex,lexDigits,lexLitChar,lines,log,logBase,lookup,map,mapM,mapM_,max,
		maxBound,maximum,maybe,min,minBound,minimum,mod,negate,not,notElem,null,numerator,odd,
		or,ord,otherwise,pi,pred,primExitWith,print,product,properFraction,putChar,putStr,putStrLn,quot,
		quotRem,range,rangeSize,read,readDec,readFile,readFloat,readHex,readIO,readInt,readList,readLitChar,
		readLn,readOct,readParen,readSigned,reads,readsPrec,realToFrac,recip,rem,repeat,replicate,return,
		reverse,round,scaleFloat,scanl,scanl1,scanr,scanr1,seq,sequence,sequence_,show,showChar,showInt,
		showList,showLitChar,showParen,showSigned,showString,shows,showsPrec,significand,signum,sin,
		sinh,snd,span,splitAt,sqrt,subtract,succ,sum,tail,take,takeWhile,tan,tanh,threadToIOResult,toEnum,
		toInt,toInteger,toLower,toRational,toUpper,truncate,uncurry,undefined,unlines,until,unwords,unzip,
		unzip3,userError,words,writeFile,zip,zip3,zipWith,zipWith3,listArray,doParse
	},
	emphstyle={[1]\color{blue}},
	emph=
	{[2]
		Bool,Char,Double,Either,Float,IO,Integer,Int,Maybe,Ordering,Rational,Ratio,ReadS,ShowS,String,
		Word8,InPacket
	},
	emphstyle={[2]\color{castanho_ulisses}},
	emph=
	{[3]
		case,class,data,deriving,do,else,if,import,in,infixl,infixr,instance,let,
		module,of,primitive,then,type,where
	},
	emphstyle={[3]\color{preto_ulisses}\textbf},
	emph=
	{[4]
		quot,rem,div,mod,elem,notElem,seq
	},
	emphstyle={[4]\color{castanho_ulisses}\textbf},
	emph=
	{[5]
		EQ,False,GT,Just,LT,Left,Nothing,Right,True,Show,Eq,Ord,Num
	},
	emphstyle={[5]\color{preto_ulisses}\textbf}
}

% Environment for displaying Haskell code
\lstnewenvironment{haskellCode}
{\vspace{-0.2cm}  \lstset{language=HaskellUlisses}}
{\smallskip}

\begin{document}
	\maketitle
	\tableofcontents
	
	\chapter*{Preface}
		This paper documents my analysis of solutions to certain problems.\
		Currently it just covers analysis of certain Project Euler problems.


	
	\chapter{Project Euler}			
	    \section{Problem One}
	        If we list all the natural numbers below 10 that are multiples of 3 or 5, we get 3, 5, 6 and 9. The sum of these multiples is 23.\
            Find the sum of all the multiples of 3 or 5 below 1000.
	            
		        \subsection{Solution One}
		            \subsubsection{Haskell}
	                	\begin{haskellCode}
{-
    sums the multiples of 3 and 5 in the list "xs" 
-}
sumIfNotMultiples :: [Int] -> Int
sumIfNotMultiples xs = sum [x | x <- xs, (mod x 3 == 0) 
                           || (mod x 5 == 0) && (mod x 15) /= 0 ]
	
	        			\end{haskellCode}

		        \subsection{Closed Solution}
		            The following function $f$ can be used to compute the sum of all the multiples of $x$ up to, but not including $y$:
		            
                    \[ f(x,y) : \frac{x \Big(\Big\lfloor \frac{(y - 1)}{x} \Big\rfloor \Big) \Big(\Big\lfloor \frac{(y - 1)}{x} \Big\rfloor + 1 \Big)}{2}\]
                    
                    Using $f$ we can compute the sums of the multiples of $3$ and $5$ up to 10 as follows:
                    
                    \[ f(3,1000) + f(5,1000) =
                        \frac{3 \Big(\Big\lfloor \frac{(10 - 1)}{3} \Big\rfloor \Big) 
                        \Big(\Big\lfloor \frac{(10 - 1)}{3} \Big\rfloor + 1 \Big)}{2}                        
                        +                        
                        \frac{5 \Big(\Big\lfloor \frac{(10 - 1)}{5} \Big\rfloor \Big) 
                        \Big(\Big\lfloor \frac{(10 - 1)}{5} \Big\rfloor + 1 \Big)}{2}  \]
		            
		            \[ \Rightarrow \frac{3(3)(4)}{2} + \frac{5(1)(2)}{2}\]

                    \[ \Rightarrow 23

    		        \subsubsection{Haskell}
    		        
	                	\begin{haskellCode}
{-
    Closed solution for problem one
-}
projectEulerOne :: Integer -> Integer -> Integer -> Integer
projectEulerOne a b n = (sumMultiples a n  + sumMultiples b n  
                            - sumMultiples (a*b) n) `div` 2
        where sumMultiples x y = x * f x y * (f x y + 1)
              f x y = ((y-1) `div` x)
                        \end{haskellCode}		            
	
\end{document}
